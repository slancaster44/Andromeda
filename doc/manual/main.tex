%! Author = slancaster
%! Date = 3/18/24

% Preamble
\documentclass{article}
\usepackage[left=3cm, right=3cm]{geometry}

\usepackage{graphicx}
\usepackage{etoolbox}
\usepackage{adjustbox}
\usepackage{array}

\makeatletter
\preto{\@verbatim}{\topsep=0pt \partopsep=0pt }
\makeatother

\setcounter{tocdepth}{2}

\title{
    Andromeda 7400\\
\large Manual}
\date{\today}
% Packages
\usepackage{amsmath}

% Document
\begin{document}
    \maketitle

    \pagebreak
    \tableofcontents
    \pagebreak

    \section{Architectural Overview}\label{subsec:archetctural-overview}
    \subsection{Design Goals \& Inspiration}\label{subsec:design-goals-&-inspiration}
    \par The Andromeda 7400 was designed to be a simple computer that could be implemented in TTL logic.
    It takes heavy design inspirations from minicomputers of the mid 20th century, such as the
    DEC PDP-8.
    \par It uses single-address instructions, where one operand resides in an accumulator register, and the other is
    sourced from memory.
    The result of an operation is almost always stored in the accumulator.
    \par Due to its design inspiration, the instruction set is highly orthogonal.
    It includes twelve instructions with seven addressing modes.

    \subsection{Accumulator}\label{subsec:accumulator}
    \par The Accumulator is a 16-bit wide register.
    It serves as the an operand and result destination for all arithmetic operations, as well as the predicate for conditional jumps.

    \subsection{Memory}\label{subsec:memory}
    \par Memory is organized into 65536 16-bit words.
    The top-most page ($FF00_{16}$ - $FFFF_{16}$) is directly addressable by most instructions, as a result it
    is used to store the most commonly referenced global variables.
    \par Memory is only addressable in word sized units; memory is not byte addressable, as is common on other machines.

    \subsection{Instruction Format}\label{subsec:instruction-format}
    \begin{center}
        \includegraphics[scale=0.40]{img/Instruction_Format}
    \end{center}
    \par Instructions are divided into three fields, the opcode, addressing mode and operand fields.
    \begin{itemize}
        \item The Opcode Field (0-4): Indicates the operation to be performed on the data
        \item The Addressing Mode Field (5-7): Determines how the operand field will be used.
        The operand field can be used as an immediate value, or as an address to some other value in memory,
        depending on the addressing mode
        \item The Operand Field (8-15): An 8-bit constant that, in combination with the Addressing Mode Field, is
        used to determine the value to use in an operation.
    \end{itemize}

    \subsection{Representing Numbers}\label{subsec:representing-numbers}
    \par All values will be stored in two's complement form.
    Addition and subtraction will both be two's complement operations.


    \subsection{Reset Sequence}\label{subsec:reset-sequence}
    \par While the machine is in reset, the reset line on the bus will be held low.
    After exiting reset, the Accumulator, Instruction Register and Program Counter will
    be set to zero.
    The machine will then begin executing code at address $0000_{16}$.


    \subsection{Invalid Instruction Trap}\label{subsec:invalid-instruction-trap}
    \par Should the machine encounter an invalid instruction, the machine will execute a JSR to address $0002_{16}$.
    That is, a pointer to the next instruction will be loaded into the accumulator, and $0002_{16}$ will be loaded
    into the program counter.
    \pagebreak

    \subsection{Addressing Modes}\label{subsec:addressing-modes}

\subsubsection{Immediate (IMM)}
\par The 8-bit operand field will be sign-extended to 16-bits.
It will then immediately be used as the secondary operand for the instruction.
\begin{verbatim}
        org(0x0000)
    def entry:
        lda.imm     -12
        hlt
\end{verbatim}`
The code above would result in `-12' being loaded into the accumulator.
Note: The 16-bit two's complement form of -12 is $1111111111110100_{2}$

\subsubsection{Memory Direct (DIR)}
\par The constant $FF00_{16}$ will be added to the 8-bit operand field.
This will yield a 16-bit value in the range $FF00_{16}$ - $FFFF_{16}$ inclusive.
This value will then be used as an address.
The value stored at that address will be used as the secondary operand for the instruction.
\begin{verbatim}
        org(0x0000)
    def entry:
        sta.dir     0x01
        hlt
\end{verbatim}
The code above would store the accumulator into the address $FF01_{16}$.

\subsubsection{Relative Direct (REL)}\label{subsec:relative-direct-(rel)}
\par The 8-bit operand field is sign-extended to 16-bits.
That value is added to the program counter (PC) to yield and address.
The contents of that address will be used as the secondary operand for the instruction.
\begin{verbatim}
        org(0x0000)
    def entry:
        lda.rel     2
        hlt
    def data:
        dw(-11)
\end{verbatim}
The code above would be result in -11 being loaded into the Accumulator.
The `lda' instruction exists at address $0000_{16}$, adding `2' to this address results in the value $0002_{16}$.
The value stored at $0002_{16}$ (`-11') is loaded into the Accumulator.

\subsubsection{Memory Indirect (IND)}
The constant $FF00_{16}$ will be added to the 8-bit operand field.
This will yield a 16-bit value in the range $FF00_{16}$ - $FFFF_{16}$ inclusive.
This value will then be used as an address.
The value stored at that address will be used as another address known as the pointer.
This value stored at this final pointer address will be used as the secondary operand for the instruction.,
\begin{verbatim}
        org(0x0000)
    def entry:
        lda.ind pointer
        hlt

        org(0xFF01)
    def pointer:
        0xD000

        org(0xD000)
    def final_value:
        -12
\end{verbatim}
The code above will result in -12 being loaded into the accumulator.
The value stored in address $FF01_{16}$ ($D000_{16}$) was used as an address to find the final value (-12).

\subsubsection{Memory Indirect \& Auto Increment (INC)}
This addressing mode behaves similarly to memory indirect addressing.
The secondary operand is obtained in the same manner.
However, after the value is obtained, the pointer will be incremented.
\begin{verbatim}
        org(0x0000)
    def entry:
        lda.inc pointer
        hlt

        org(0xFF01)
    def pointer:
        0xD000

        org(0xD000)
    def final_value:
        -12
\end{verbatim}
The code above will result in -12 being loaded into the accumulator.
However, after the instruction has completed execution, pointer will be incremented to $D001_{16}$

\subsubsection{Auto Decrement \& Memory Indirect (DEC)}
This addressing mode behaves similarly to memory indirect addressing.
Before the instruction is executed, the pointer will be decremented.
Then the secondary operand is obtained in the same manner as it is in simple Memory Indirect addressing.
\begin{verbatim}
        org(0x0000)
    def entry:
        lda.dec pointer
        hlt

        org(0xFF01)
    def pointer:
        0xD000

        org(0xCFFF)
    def final_value:
        13
        -12
\end{verbatim}
The code above will result in 13 being loaded into the accumulator.
Pointer will have been decremented to $CFFF_{16}$ so the `13' could be addressed indirectly
    \pagebreak
    \subsection{LDA}\label{subsec:lda}
    \subsubsection{Description}
    $Accumulator \leftarrow Operand Value$
    \par Load the value indicated by the operand and addressing mode fields into the Accumulator.

    \subsubsection{Encoding}

    \subsubsection{Process}
    \begin{enumerate}
        \item
    \end{enumerate}

    \subsubsection{Example}
    \begin{verbatim}
        org(0x0000)
    def entry:
        lda.dir     data

        org(0xFF01)
    def data:
        dw(-11)
    \end{verbatim}
    \par The above code would result in `-11' being loaded into the Accumulator.


\subsection{STA}\label{subsec:sta}
    \subsubsection{Description}
    $if\ (AddressingMode \neq IMM)\ \{ *OperandValue \leftarrow Accumulator \}$ \\
    $else\ \{ *Accumulator \leftarrow OperandValue \}$
    \par If the addressing mode is not `imm,' then store the contents of the accumulator into the address indicated by
    the operand value.
    If the addressing mode is `imm,' the operand value is loaded into the address indicated by the Accumulator.

    \subsubsection{Encoding}
    \subsubsection{Process}
    \begin{enumerate}
        \item
    \end{enumerate}

    \subsubsection{Example}
    \begin{verbatim}
    \end{verbatim}

\subsection{ADD}\label{subsec:add}
    \subsubsection{Description}
    $Accumulator \leftarrow Accumulator + OperandValue$
    \par Add the contents of the Accumulator, and the value indicated by the operand and addressing fields.
    Store the result in the Accumulator.

    \subsubsection{Encoding}
    \subsubsection{Process}
    \begin{enumerate}
        \item
    \end{enumerate}

    \subsubsection{Example}
    \begin{verbatim}
    \end{verbatim}

\subsection{SUB}\label{subsec:sub}
    \subsubsection{Description}
    $Accumulator \leftarrow Accumulator - OperandValue$
    \par Subtract the value indicated by the operand and addressing fields from the contents of the Accumulator.
    Store the result in the Accumulator.

    \subsubsection{Encoding}
    \subsubsection{Process}
    \begin{enumerate}
        \item
    \end{enumerate}

    \subsubsection{Example}
    \begin{verbatim}
    \end{verbatim}

\subsection{XOR}\label{subsec:xor}
    \subsubsection{Description}
    $Accumulator \leftarrow Accumulator \oplus OperandValue$
    \par Exclusive or the contents of the Accumulator, and the value indicated by the operand and addressing fields.
    Store the result in the Accumulator.

    \subsubsection{Encoding}
    \subsubsection{Process}
    \begin{enumerate}
        \item
    \end{enumerate}

    \subsubsection{Example}
    \begin{verbatim}
    \end{verbatim}

\subsection{NND}\label{subsec:nand}
    \subsubsection{Description}
    $Accumulator \leftarrow \overline{Accumulator \land OperandValue}$
    \par Nand the contents of the Accumulator, and the value indicated by the operand and addressing fields.
    Store the result in the Accumulator.

    \subsubsection{Encoding}
    \subsubsection{Process}
    \begin{enumerate}
        \item
    \end{enumerate}

    \subsubsection{Example}
    \begin{verbatim}
    \end{verbatim}

\subsection{JMP}\label{subsec:jmp}
    \subsubsection{Description}
    $PC \leftarrow OperandValue$
    \par Load the value indicated by the operand and addressing fields into the program counter.
    \subsubsection{Encoding}
    \subsubsection{Process}
    \begin{enumerate}
        \item
    \end{enumerate}

    \subsubsection{Example}
    \begin{verbatim}
    \end{verbatim}

\subsection{JSR}\label{subsec:jsr}
    \subsubsection{Description}
    $Accumulator \leftarrow PC$; $PC \leftarrow OperandValue$
    \par Load the program counter into the Accumulator.
    Load the operand value into the program counter.
    \subsubsection{Encoding}
    \subsubsection{Process}
    \begin{enumerate}
        \item
    \end{enumerate}

    \subsubsection{Example}
    \begin{verbatim}
    \end{verbatim}

\subsection{JNS}\label{subsec:jns}
    \subsubsection{Description}
    $if\ (Accumulator \geq 0)\ \{ PC \leftarrow OperandValue \}$
    \par If the contents of the Accumulator are greater than, or equal to, zero, load the operand value into the PC\@.
    \subsubsection{Encoding}
    \subsubsection{Process}
    \begin{enumerate}
        \item
    \end{enumerate}

    \subsubsection{Example}
    \begin{verbatim}
    \end{verbatim}

\subsection{JNZ}\label{subsec:jnz}
    \subsubsection{Description}
    $if\ (Accumulator \neq 0)\ \{ PC \leftarrow OperandValue \}$
    \par If the contents of the Accumulator are not equal to zero, load the operand value into the PC\@.
    \subsubsection{Encoding}
    \subsubsection{Process}
    \begin{enumerate}
        \item
    \end{enumerate}

    \subsubsection{Example}
    \begin{verbatim}
    \end{verbatim}

\subsection{HALT}\label{subsec:halt}
    \subsubsection{Description}
    $HFF \leftarrow 1$
    \par Stop the system clock by setting the halt flip-flop.
    \subsubsection{Encoding}
    \subsubsection{Process}
    \begin{enumerate}
        \item
    \end{enumerate}

    \subsubsection{Example}
    \begin{verbatim}
    \end{verbatim}

\subsection{NOP}\label{subsec:nop}
    \subsubsection{Description}
    $ -- $
    \par Continue to the next instruction without changing the machine's state.
    \subsubsection{Encoding}
    \subsubsection{Process}
    \begin{enumerate}
        \item
    \end{enumerate}

    \subsubsection{Example}
    \begin{verbatim}
    \end{verbatim}
    \section{Control Algorithm}\label{sec:abstract-description}
\subsection{Block Diagram}\label{subsec:block-diagram}
\begin{center}
    \includegraphics[scale=0.58]{img/Andromeda-Block Diagram.drawio}
\end{center}

\subsection{Instruction Flow}\label{subsec:state-machine-diagrams}
\par This section outlines the algorithm each instruction/addressing mode pair will execute.
The algorithm is laid out in terms of what steps will be executed for every rising and falling edge of the system clock.
Each step will correspond to a single micro-instruction.

\pagebreak
\subsubsection{\texttt{lda.imm}}
\subsubsection{\texttt{lda.dir}}
\subsubsection{\texttt{lda.rel}}
\subsubsection{\texttt{lda.off}}
\subsubsection{\texttt{lda.ind}}
\subsubsection{\texttt{lda.inc}}
\subsubsection{\texttt{lda.dec}}

\subsubsection{\texttt{sta.imm}}
\subsubsection{\texttt{sta.dir}}
\subsubsection{\texttt{sta.rel}}
\subsubsection{\texttt{sta.off}}
\subsubsection{\texttt{sta.ind}}
\subsubsection{\texttt{sta.inc}}
\subsubsection{\texttt{sta.dec}}

\subsubsection{\texttt{add.imm}}
\subsubsection{\texttt{add.dir}}
\subsubsection{\texttt{add.rel}}
\subsubsection{\texttt{add.off}}
\subsubsection{\texttt{add.ind}}
\subsubsection{\texttt{add.inc}}
\subsubsection{\texttt{add.dec}}

\subsubsection{\texttt{xor.imm}}
\subsubsection{\texttt{xor.dir}}
\subsubsection{\texttt{xor.rel}}
\subsubsection{\texttt{xor.off}}
\subsubsection{\texttt{xor.ind}}
\subsubsection{\texttt{xor.inc}}
\subsubsection{\texttt{xor.dec}}


    \section{Hardware}\label{sec:hardware}

\end{document}