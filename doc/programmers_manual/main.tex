%! Author = slancaster
%! Date = 3/18/24

% Preamble
\documentclass[11pt]{article}
\usepackage{graphicx}

\title{
    Andromeda 7400\\
\large Programmer's Manual}
\date{February 2024}
% Packages
\usepackage{amsmath}

% Document
\begin{document}
    \maketitle

    \pagebreak
    \tableofcontents
    \pagebreak

    \section{Architectural Overview}\label{subsec:archetctural-overview}
    \subsection{Design Goals \& Inspiration}\label{subsec:design-goals-&-inspiration}
    \par The Andromeda 7400 was designed to be a simple computer that could be implemented in TTL logic.
    It takes heavy design inspirations from minicomputers of the mid to late 20-th century, such as the
    DEC PDP-8.
    \par It uses single-address instructions, where one operand resides in the accumulator, and the other is
    sourced from memory.
    The result of an operation is almost always stored in the accumulator.
    \par The instruction set is slightly more sophisticated than the PDP-8, and is highly orthogonal.
    It includes twelve instructions with five addressing modes.

    \subsection{Accumulator}\label{subsec:accumulator}
    \par The Accumulator is a 16-bit wide register.
    It serves as the an operand to all arithmetic operations, as well as the predicate for conditional jumps.

    \subsection{Memory}\label{subsec:memory}
    \par Memory is organized into 65536 16-bit words.
    The top-most page ($FF00_{16}$ - $FFFF_{16}$) is directly addressable by most instructions, as a result it
    is used to store the most commonly referenced global variables.
    \par Memory is only addressable in word sized units; memory is not byte addressable, as is common on other machines
    \subsection{Instruction Format}\label{subsec:instruction-format}
    \par Instructions are divided into three fields.
    \begin{itemize}
        \item The Opcode Field (0-4): Indicates the operation to be performed on the data
        \item The Addressing Mode Field (5-7): The Addressing Mode Field determines how the operand field will be used.
        The operand field can be used as an immediate value, or as an address to some other value in memory,
        depending on the addressing mode
        \item The Operand Field (8-15): An 8-bit constant that, in combination with the Addressing Mode Field, is
        used to determine the second value to use in an operation (the first value being whatever is stored in the Accumulator).
    \end{itemize}

    \subsection{Representing Numbers}\label{subsec:representing-numbers}
    \par All values will be stored in two's complement form.
    Addition and subtraction will both be two's complement operations.

    \subsection{Addressing Modes}\label{subsec:addressing-modes}

    \subsubsection{Immediate (IMM)}
    \par The 8-bit operand field will be sign-extended to 16-bits.
    It will then immediately be used as the secondary operand for the instruction.
    \begin{verbatim}
        org(0x0000)
    def entry:
        lda.imm     -12
        hlt
    \end{verbatim}`
    The code above would result in `-12' being loaded into the accumulator.
    Note: The 16-bit two's complement form of -12 is $1111111111110100_{8}$

    \subsubsection{Memory Direct (DIR)}
    \par The constant $FF00_{16}$ will be added to the 8-bit operand field.
    This will yield a 16-bit value in the range $FF00_{16}$ - $FFFF_{16}$ inclusive.
    This value will then be used as an address.
    The value stored at that address will be used as the secondary operand for the instruction.
    \begin{verbatim}
        org(0x0000)
    def entry:
        sta.dir     0x01
        hlt
    \end{verbatim}
    The code above would store the accumulator into the address $FF01_{16}$.

    \subsubsection{Memory Indirect (IND)}
    The constant $FF00_{16}$ will be added to the 8-bit operand field.
    This will yield a 16-bit value in the range $FF00_{16}$ - $FFFF_{16}$ inclusive.
    This value will then be used as an address.
    The value stored at that address will be used as another address known as the pointer.
    This value stored at this final pointer address will be used as the secondary operand for the instruction.,
    \begin{verbatim}
        org(0x0000)
    def entry:
        lda.ind pointer
        hlt

        org(0xFF01)
    def pointer:
        0xD000

        org(0xD000)
    def final_value:
        -12
    \end{verbatim}
    The code above will result in -12 being loaded into the accumulator.
    The value stored in address $FF01_{16}$ ($D000_{16}$) was used as an address to find the final value (-12).

    \subsubsection{Memory Indirect \& Auto Increment (INC)}
    This addressing mode behaves similarly to memory indirect addressing.
    The secondary operand is obtained in the same manner.
    However, after the value is obtained, the pointer will be incremented.
    \begin{verbatim}
        org(0x0000)
    def entry:
        lda.inc pointer
        hlt

        org(0xFF01)
    def pointer:
        0xD000

        org(0xD000)
    def final_value:
        -12
    \end{verbatim}
    The code above will result in -12 being loaded into the accumulator.
    However, after the instruction has completed execution, pointer will be incremented to $D001_{16}$

    \subsubsection{Memory Indirect \& Auto Decrement (DEC)}
    This addressing mode behaves similarly to memory indirect addressing.
    The secondary operand is obtained in the same manner.
    However, after the value is obtained, the pointer will be decremented.
    \begin{verbatim}
        org(0x0000)
    def entry:
        lda.dec pointer
        hlt

        org(0xFF01)
    def pointer:
        0xD000

        org(0xD000)
    def final_value:
        -12
    \end{verbatim}
    The code above will result in -12 being loaded into the accumulator.
    However, after the instruction has completed execution, pointer will be decremented to $CFFF_{16}$

    \subsection{Block Diagram}\label{subsec:block-diagram}
    \pagebreak

    \section{Instruction Set}
    \subsection{LDA}
    \subsection{STA}
    \subsection{ADD}
    \subsection{SUB}
    \subsection{XOR}
    \subsection{NAND}
    \subsection{JMP}
    \subsection{JSR}
    \subsection{JNS}
    \subsection{JNZ}
    \subsection{HALT}
    \subsection{NOP}

\end{document}